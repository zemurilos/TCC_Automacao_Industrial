\chapter{Introdução}
\label{cap:01}
A energia tem se tornado cada vez mais importante no mundo atual, assim como a busca de meios de energias renováveis e sustentáveis. Entre as principais formas de geração de energia no Brasil são citáveis as energias eólica e solar, que com o avanço da tecnologia têm se tornado cada vez mais acessível. No Brasil, há grande potencial de geração de energia solar, devido a sua posição geográfica, possuindo diversas áreas tropicais, favorecendo assim a geração fotovoltaica. Entre as formas de energia solar, o uso da energia fotovoltaica tem se tornado mais popular. Entretanto, para garantir que um sistema que utilize energia fotovoltaica seja viável é necessário o uso de algumas tecnologias que permitam uma geração eficiente, como o uso MPPT, \textit{Maximum Power Point Tracker}, ou Rastreador de Máxima Potência, de forma a garantir que seja extraída a maior potência do sistema fotovoltaico.

\indent		Tendo este aspecto em mente, este trabalho tem como premissa desenvolver um protótipo de um traçador de curva I-V portátil para painéis fotovoltaicos, podendo assim detectar anomalias em painéis durante a geração, como sombreamentos, curto circuitos e outros defeitos sobre o módulo ou painel em análise. 


\section{Justificativa}

No Brasil, cerca de 90,9\% da geração de energia total foram por meio de energias renováveis, sendo a matriz hidráulica se mantem dominante com 81,0\%, e tendo 8,2\% as usinas eólicas e 0,6\% as solares, em janeiro de 2019 de acordo com o Boletim de Monitoramento do Sistema Elétrico, divulgado pelo Ministério de Minas e Energias.%ADICIONAR REFERÊNCIA

\indent	Nota-se que a energia solar ainda está se popularizando no Brasil, entretanto segundo a Organização das Nações Unidas(ONU), os investimentos focados em energia solar já ultrapassam a casa dos US\$ 160 bilhões, se tornando cada vez mais importantes em um contexto de geração de energia sustentável. Dito isso, há a necessidade de engajar o uso e conhecimento deste meio de geração de energia, desta maneira, permitindo ao país a diversificação de suas fontes de geração elétrica, de maneira a permitir maior flexibilidade e uma menor dependência a apenas um meio.
Realidade a qual pode gerar diversas consequências em caso de problemas ou falta na geração a partir desse meio, como aumento das taxas pagas sobre o consumo de energia, desencadeando diversos problemas econômicos, sociais, e estruturais sobre um país. Entretanto, ao estimular o uso da energia solar é possível descentralizar as fontes de geração de energia por meio de geração limpa, sustentável, e viável, gerando diversas oportunidades de trabalho e estudo.%ADICIONAR REFERÊNCIA


\section{Objetivos}

\subsection{Objetivo Geral}

Analisar o comportamento de painéis fotovoltaicos por meio do uso de sua curva I-V.

\subsection{Objetivos Específicos}
\begin{itemize}
	\item Desenvolver um sistema traçador de curva I-V de baixo custo;
	\item Comparar curvas I-V durante diferentes níveis de irradiação solar em painéis fotovoltaicos;%diferentes painéis fotovoltaicos;
	%\item Valorizar o uso de sistemas fotovoltaicos para estudo e uso em faculdades e empresas no Brasil;
	%\item Categorizar usos diversos de sistemas traçadores de curva I-V para diferentes aplicações para estudo ou comercialmente.

\end{itemize}

\section{Metodologia}

A metodologia utilizada durante a realização do trabalho tem como base o tipo de pesquisa tecnológica exploratória, de maneira a descrever e desenvolver a curva I-V de um painel fotovoltaico e seu uso em estudos ou uso comercial. Deste modo foram realizados os seguintes passos:1.Análise bibliográfica; 2.Desenvolvimento de um sistema protótipo; 3.Teste do circuito; 4.Análise dos dados coletados; 5.Teste para diferentes quantidades de conjuntos de valores. 6.Comparação e análise dos dados coletados, gráficos gerados e curva teórica.
