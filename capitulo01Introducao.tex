\chapter{Introdução}
\label{cap:01}
A energia tem se tornado cada vez mais importante no mundo atual, assim como a busca de meios de energias renováveis e sustentáveis. Entre as principais formas de geração de energia no Brasil são citáveis as energias eólica e solar, que com o avanço da tecnologia têm se tornado cada vez mais acessível. No Brasil, há grande potencial de geração de energia solar, devido a sua posição geográfica, próximo a linha do Equador. Entre as formas de energia solar, o uso da energia fotovoltaica têm se tornado mais popular. Entretanto, para garantir que um sistema que utilize energia fotovoltaica seja viável é necessário o uso de alguns processos que permitam uma geração eficiente, como o uso MPPT, \textit{Maximum Power Point Tracker}, ou Rastreador de Máxima Potência, de forma a garantir que o sistema dê a maior potência possível em um determinado período.

\indent		Tendo este aspecto em mente, este trabalho terá como premissa o desenvolvimento de uma metodologia de análise de sistemas fotovoltaicos a partir do uso do MPPT, de maneira a garantir e viabilizar o uso de sistemas fotovoltaicos que sejam compactos, eficazes e baratos.
Logo, através de um sistema que gera dados que disponibilizem o modo de atuação de um sistema fotovoltaico, é possível encontrar o ponto de máxima potência e gerar diversos gráficos a respeito do mesmo, assim permitindo uma análise completa e a comparação com diversos sistemas.


\section{Justificativa}

No Brasil, cerca de 81,9\% da capacidade de geração de energia e 87,8\% da produção total foram por meio de energias renováveis, sendo a matriz hidráulica ainda dominante com 63,7\%, e tendo 8,1\% as usinas eólicas e 1\% as solares, em junho de 2018 de acordo com o Boletim de Monitoramento do Sistema Elétrico, divulgado pelo Ministério de Minas e energias.%ADICIONAR REFERÊNCIA

\indent	Nota-se que a energia solar ainda está se popularizando no Brasil, entretanto segundo a Organização das Nações Unidas(ONU), os investimentos focados em energia solar já ultrapassam a casa dos US\$ 160 bilhões, se tornando cada vez mais importantes em um contexto de geração de energia sustentável. Dito isso, há a necessidade de engajar o uso e conhecimento deste meio de geração de energia, desta maneira, permitindo ao país a diversificação de suas fontes de geração elétrica, de maneira a permitir maior flexibilidade e uma menor dependência a apenas um meio.
Realidade a qual pode gerar diversas consequências em caso de problemas ou falta na geração a partir desse meio, como aumento das taxas pagas sobre o consumo de energia, desencadeando diversos problemas econômicos, sociais, e estruturais sobre um país. Entretanto, ao estimular o uso da energia solar é possível descentralizar as fontes de geração de energia por meio de um meio de geração limpo, sustentável, e viável, gerando diversas oportunidades de geração de trabalho e estudo.%ADICIONAR REFERÊNCIA


\section{Objetivos}

\subsection{Objetivo Geral}

Analisar o comportamento de painéis fotovoltaicos por meio do uso de sua curva I-V.

\subsection{Objetivos Específicos}
\begin{itemize}
	\item Desenvolver um sistema traçador de curva I-V de baixo custo;
	\item Comparar curvas I-V durante diferentes níveis de irradiação solar em diferentes painéis fotovoltaicos;
	\item Valorizar o uso de sistemas fotovoltaicos para estudo e uso em faculdades e empresas ao redor do Brasil;
	\item Categorizar usos diversos de sistemas traçadores de curva I-V para diferentes aplicações para estudo ou comercialmente.

\end{itemize}

\section{Metodologia}

A metodologia utilizada durante a realização do trabalho tem como base o tipo de pesquisa tecnológica exploratória, de maneira a descrever e desenvolver a curva I-V de um painel fotovoltaico e seu uso em estudos ou uso comercial. Deste modo foram realizados os seguintes passos:1.Análise bibliográfica; 2.Desenvolvimento de um sistema protótipo; 3.Teste do circuito; 4.Análise dos dados coletados; 5.Teste para diferentes quantidades de conjuntos de valores. 6.Comparação e análise dos dados coletados, gráficos gerados e curva teórica.
