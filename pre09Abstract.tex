\noindent{SILVA, M. F.; LOPES, V. H. D. \textbf{Traçador de curva I-V para painéis fotovoltaicos}. 2019. Trabalho de conclusão de curso (Superior em Tecnologia em Automação Industrial) – Instituto Federal de Educação, Ciência e Tecnologia de São Paulo, Câmpus Guarulhos, Guarulhos. 2019.}
%\noindent{SOBRENOME, Prenome. \textbf{Título do trabalho de TCC colocado em negrito:} subtítulo (se houver). Ano da defesa. Tipo de documento (Grau e vinculação acadêmica) – Instituição, Local. Ano da entrega.}

% resumo em inglês
\begin{resumo}[Abstract]
	\begin{otherlanguage*}{english}
		
	After the 2010 water crisis, there was an increase in investment in research on clean and renewable energy, among them is the photovoltaic solar energy. This paper presents a low cost measurement instrument, capable of measuring the IxV (current per voltage) curve of a photovoltaic panel. The importance of obtaining these measures comes from the need to identify possible problems, defects and / or abnormalities on the panel. For the elaboration of this instrument of measurement, an electronic resistance was used to simulate the resistance variation in the panel terminals, Arduino prototyping platform with an Atmega328p microcontroller for control, processing and storage of data. After assembling the circuit, several solar measurement tests were carried out in order to verify the change of the curve in different levels of solar radiation power, to analyze distortions when part of the panel was shaded, and to measure the power decrease generated due to the heating of the panel surface. After testing and the data analysis, it was found that the project has functionality, focusing on its use in the academic area, encouraging new researchers to develop technologies in the area of solar energy and in the industrial area. The project has demonstrated great importance in the identification of any abnormality of both circuit and panel surface.

		
		\vspace{\onelineskip}
		 
		\textbf{Keywords}: Keyword 1. Keyword 2. Keyword 3. Keyword n.
	\end{otherlanguage*}
\end{resumo} 