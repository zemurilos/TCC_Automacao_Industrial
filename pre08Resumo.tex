%\noindent{SOBRENOME, Prenome. \textbf{Título do trabalho de TCC colocado em negrito:} subtítulo (se houver). Ano da defesa. Tipo de documento (Grau e vinculação acadêmica) – Instituição, Local. Ano da entrega.}

\noindent{SILVA, M. F.; LOPES, V. H. D. \textbf{Traçador de curva I-V para painéis fotovoltaicos}. 2019. Trabalho de conclusão de curso (Superior em Tecnologia em Automação Industrial) – Instituto Federal de Educação, Ciência e Tecnologia de São Paulo, Câmpus Guarulhos, Guarulhos. 2019.}


\setlength{\absparsep}{18pt} % ajusta o espaçamento dos parágrafos do resumo
\begin{resumo}
	
	Após a crise hídrica de 2010, houve um aumento de investimento em pesquisa de fontes de energia limpa e renováveis, dentre elas, a fotovoltaica proveniente da irradiância solar. O trabalho apresentado, traz um instrumento de medida de baixo custo capaz de traçar a curva IxV (corrente por tensão) de um painel fotovoltaico e tem como objetivo analisar o comportamento de um painel fotovoltaico. A importância de obter essas medidas vem da necessidade de identificar possíveis problemas, defeitos e/ou anormalidades no painel. Para a elaboração deste instrumento de medida, foi utilizado uma carga eletrônica para simular a variação de resistência nos terminais do painel, e a plataforma de prototipagem Arduino com um microcontrolador Atmega328p para controle, processamento e armazenamento de dados. Após a montagem do circuito, efetuou-se diversos testes de medição solar, a fim de constatar a alteração da curva em diferentes níveis de potência de irradiação solar, analisar distorções quando parte do painel foi sombreada e  medir a diminuição da potência gerada devido ao aquecimento da superfície do painel. Após os teste e a análise de dados, constatou-se a funcionalidade do projeto, tendo em foco o emprego dele na área acadêmica, impulsionando futuros pesquisadores a desenvolver tecnologias na área de geração de energia solar,  e na área industrial sendo de grande importância na identificação de qualquer anormalidade tanto de circuito quanto na superfície do painel.
	
	\vspace{\onelineskip}
	
	\textbf{Palavras-chave}: Fotovoltaico. Traçador. IV. Carga. Eletrônica.
\end{resumo}