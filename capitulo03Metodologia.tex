\chapter{Materiais e Métodos}
\label{cap:03}

Para a prototipagem do traçador I-V houve a necessidade de utilizar as seguintes etapas:
\begin{itemize}
	\item Integração entre os componentes do protótipo sendo eles Arduino;
	\item Carga eletrônica;
	\item Painel solar;
	\item Sensor ADS1115;
	\item Cartão SD.
\end{itemize}

\subsection{Arduino}
A plataforma Arduino foi utilizado de maneira centralizar o controle de todos os periféricos necessários para a prototipagem. Foi utilizado as saídas PWM, Barramento I2C e SPI.

IMAGEM que será adicionada
%Imagem ARduino+I2C+SPI

\subsection{Carga Eletronica}
A carga eletrônica permitiu a variação entres valores de Corrente e Tensão, fator extremamente importante para o traçador.

Imagem Carga eletrônica



\subsection{Métodos}
Durante os experimentos foram necessários fixar alguns aspectos:
\begin{itemize}
\item Irradiância Solar
\item Sensores Precisos
\end{itemize}
...