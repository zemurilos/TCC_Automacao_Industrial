\chapter{Revisão da Literatura}
\label{cap:02}

=======
Há grande importância e interesse no uso da curva I-V de sistemas fotovoltaicos, para gerenciamento e funcionamento destes sistemas, assim como descreve Schill, Brachmann e Koehl (2015), no \textit{Fraunhofer Institute for Solar Energy Systems ISE}, faz-se o uso da curva I-V para verificação da perfomace dos arranjos fotovoltaicos durantes teste em ambientes abertos para verificação da durabilidade de materiais para o uso em sistemas fotovoltaicos. De acordo com os autores é verificado e monitorado a curva a cada 10 minutos. O uso da curva I-V permitiu a conclusão em teste onde os paineis foram sujos e permaneceram, durante o teste foi verificados uma diminuição de até 20% dos valores iniciais de eficiência.
Assim como aponta Schill, Brachmann e Koehl (2015), o elemento central para a geração da curva I-V é a carga eletronica, o qual tem como principal função simular diferentes cargas para o painel fotovoltaico, de maneira a permitir verificar seu comportamento, assim também utilizado por Aliaga et al. (2016), na criação de sistemas que garatam a máxima potência global de um arranjo fotovoltaico.
O sistema de carga eletronica com Mosfet permite a rapida variação de carga sobre o painel fotovoltaico de maneira a permitir a construção da curva I-V característica do painel em determinado instante, desta maneira diminuir a diferença de geração devidos a meios externos como nuvens ou variações climaticas, (WILLOUGHBY; OSINOWO, 2018).

Texto da revisão da literatura, dividido em seções e subseções.


Este é um exemplo de como usar figuras. Referência cruzada: Figura~\ref{fig:exemplo}

\FloatBarrier
\begin{figure}[!htbp]
	\centering
	\caption{Exemplo de figura}
	%scale redimensiona a figura.
	%1.5 = 150% do tamanho original
	%1 = 100% do tamanho original
	%0.20 = 20% do tamanho original
	\includegraphics[scale=1.5]{imagens/exemploFigura}
	\\\textbf{Fonte:} Elaborada pelo autor
	\label{fig:exemplo}
\end{figure}
\FloatBarrier


Este é um exemplo de como usar tabelas. Referência cruzada: Tabela~\ref{tab:exemplo}

\FloatBarrier
\begin{table}[!htbp]
\centering
\caption{Exemplo de tabela de 2 colunas}
	\begin{tabular}{ c | c }
		\hline
		\textbf{Coluna 1} & \textbf{Coluna 2} \\ \hline
		Dado 1a           & Dado 1b           \\ \hline
		Dado 2a           & Dado 2b           \\ \hline
		Dado 3a           & Dado 3b           \\ \hline
		Dado 4a           & Dado 4b           \\ \hline
	\end{tabular}
	\\ \vspace{0.2cm}
	\textbf{Fonte:} Elaborada pelo autor
	\label{tab:exemplo}
\end{table}
\FloatBarrier


Este é um exemplo de como usar quadros. Referência cruzada: Quadro~\ref{qua:exemplo}

\FloatBarrier
\begin{quadro}[!htbp]
	\centering
	\caption{Exemplo de quadro}
	\includegraphics[scale=.7]{imagens/exemploQuadro}
	\\\textbf{Fonte:} Elaborada pelo autor
	\label{qua:exemplo}
\end{quadro}
\FloatBarrier


Este é um exemplo de como usar equações. Referência cruzada: Equação~\ref{eq:exemplo}

\begin{equation}
\sum_{i=1}^{n} = \frac{n(n+1)}{2}
\label{eq:exemplo}
\end{equation}


Exemplo de inserção de lista de código fonte (\textbf{\textcolor{red}{não use acentos no código!}}):

\lstinputlisting[language=Java]{fontes/ClasseExemplo.java} 



Este é um exemplo de como inserir texto sem formatação (ambiente verbatim):

\begin{verbatim}
	Texto sem formatação, como espaçamento igual.
\end{verbatim}


Exemplo de lista de itens:

\begin{itemize}
	\item \textbf{Item 1:} texto...;
	\item \textbf{Item 2:} texto...;
    \begin{itemize}
            \item \textbf{Subitem:} texto...;
            \item \textbf{Subitem:} texto...;
            \item \textbf{Subitem:} texto...;
        \end{itemize}
	\item \textbf{Item 3:} texto...;
	\item \textbf{Item n:} texto....
\end{itemize}


Exemplo de lista numerada:

\begin{enumerate}
	\item \textbf{Item:} texto...;
	\item \textbf{Item:} texto...;
    \begin{enumerate}
        \item \textbf{Subitem:} texto...;
        \item \textbf{Subitem:} texto...;
        \item \textbf{Subitem:} texto...;
    \end{enumerate}
	\item \textbf{Item:} texto...;
	\item \textbf{Item:} texto....
\end{enumerate}


Exemplos de comandos para texto e referências:

\begin{itemize}
	\item Para iniciar um novo parágrafo, basta deixar uma linha em branco no código fonte;
	\item Não force o compilador a pular mais de uma linha, pois terá influência negativa na composição do documento;
	\item Sempre deixe o \LaTeX\ realizar a formatação de parágrafos e posicionamento de elementos;
	\item Utilização de aspas simples (abertura \verb|`|, fechamento \verb|'|): `Texto entre aspas simples';
	\item Utilização de aspas duplas (abertura \verb|``|, fechamento \verb|''|): ``Texto entre aspas duplas'';
	\item Negrito (comando \verb|\textbf|): \textbf{texto em negrito};
	\item Itálico (comando \verb|\textit|): \textit{texto em itálico};
	\item Sublinhado (comando \verb|\underline|): \underline{texto sublinhado};
	\item Negrito e itálico (usar comandos juntos): \textbf{\textit{texto em negrito e itálico}};
	\item Alterar cor do texto (comando \verb|\textcolor{cor}{texto}|):
	\begin{itemize}
		\item Exemplo \verb|\textcolor{red}{texto}|: \textcolor{red}{texto vermelho};
		\item Exemplo \verb|\textcolor[RGB]{255, 102, 0}|: \textcolor[RGB]{255, 102, 0}{texto laranja};
		\item Exemplo \verb|\textcolor[HTML]{006AD7}|: \textcolor[HTML]{006AD7}{texto azul};
	\end{itemize}
	\item Ambiente matemático inline (comando \verb|$ expressão $|): $s = x^2-2x +1$;
	\item Referência normal (comando \verb|\cite|):
	\begin{itemize}
		\item \cite{Agaisse1995};
		\item \cite{Abedi2014};
		\item \cite{BtNomenclature2016};
	\end{itemize}
	\item Referência normal com mais de uma obra (comando \verb|\cite|):
	\begin{itemize}
		\item \cite{Agaisse1995, Abedi2014};
		\item \cite{Nelson2014, BtNomenclature2016, AgapitoTenfen2014};
	\end{itemize}
	\item Referência nome e ano (comando \verb|\citeauthorandyear|):
	\begin{itemize}
		\item \citeauthorandyear{Agaisse1995};
		\item \citeauthorandyear{Abedi2014};
		\item \citeauthorandyear{BtNomenclature2016};
	\end{itemize}
\end{itemize}


Exemplo 1 de referência direta:

\begin{citacao}
	Os 20 aminoácidos usualmente encontrados como resíduos em proteínas contém um grupo $\alpha$-carboxil, um grupo $\alpha$-amino e um grupo R distinto substituído no átomo de carbono $\alpha$. O átomo de carbono $\alpha$ de todos os aminoácidos, com exceção da glicina, é assimétrico e, portanto, os aminoácidos podem existir em pelo menos duas formas estereoisoméricas. Somente os estereoisômeros L, com uma configuração relacionada à configuração absoluta da molécula de referência L-gliceraldeído, são encontrados em proteínas \cite[p. 81]{Nelson2014}
\end{citacao}

Exemplo 2 de referência direta:

\begin{citacao}
	\textit{These various insecticidal proteins are synthesized during the stationary phase and accumulate in the mother cell as a crystal inclusion which can account for up to 25\% of the dry weight of the sporulated cells. The amount of crystal protein produced by a B. thuringiensis culture in laboratory conditions (about 0.5 mg of protein per ml) and the size of the crystals (24) indicate that each cell has to synthesize $10^6$ to $2 \times 10^6$ $\delta$-endotoxin molecules during the stationary phase to form a crystal} \cite[p. 1]{Agaisse1995}
\end{citacao}

Exemplo de nota de rodapé\footnote{Essa é uma nota de rodapé!}.
